\subsection{Education \& Open Collaboration}
\label{classmotivations}

The {\it Open Collaboration and Peer-Production class (i290m OCPP)} is designed to combine hands-on exploration of the theory and practice of open online collaboration with students engaging multi-disciplinary literature about collaboration, joining and contributing to an existing open collaboration project involving programming or not. It recognizes that joining and getting integrated in project communities can be difficult but also greatly rewarding, and has been designed to facilitate integration and at the same time acquire the right tools to understand and possibly monitor how communities works. The open collaboration approach is remote from usual academic teaching settings and requires adaptation \todo{put more emphasis on the required adaptation : peer-review, task self-selection etc}. The i290m OCPP class offers the opportunity to discover and smoothly transition to this world. It is aligned with similar requirements increasingly requested by the industry, but also by the emerging field of reproducible data science, which has {\it de facto} embraced the open source model. Ironically open source software development  was initially inspired by academic research\cite{bezroukov1999oss}.

Besides attending and actively interacting in class as well as participating to an open collaboration project  \cite{classweb2013}, students have to produce two main deliverables : (i) guided individual assignments as blog posts on their community immersion structure (see Section \ref{qualitative_reporting}) and (ii) a collective report for students must pretty much self-organize to find a topic, pool their various experiences, compile and analyze data, discuss results and conclude. 

The report is an open collaboration project itself with the class being the community. The i290m OCPP class is designed to let as many contribution options as possible following {\it task self-selection} as well as many {\it peer-review} opportunities following the peer-production principles \cite{benkler2002}. Following the intrinsic motivation component of open collaboration students choose the open collaboration project(s) they want to contribute to, the way to address assignments that are pretty open, and how to engage in the collective report. In the end the success of the report relies on how students can collectively reuse what they have learned during the class and with the accumulated field experience. 


