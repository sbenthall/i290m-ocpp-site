\section{Conclusion}

This report originates from the course entitled Open Collaboration and Peer-Production (i290m-ocpp) \cite{classweb2013} at UC Berkeley I School , with the main goals to get {\it hands-on} experience of joining and contributing to open collaboration projects, and to understand the underpinning incentive mechanisms and contribution dynamics. Joining open source projects is not as simple as one might expect: multiple technical, incentive, social and emotional issues as well as imperfect communication between communities and newbies might prevent efficient joining. \\

\noindent The experience acquired in the class guided our field exploration of joining scripts and how they are influenced by organization structures. Acquiring and maintaining a critical mass of developers as well as solving governance problems are among the most critical organizational issues in open collaboration. Here, we have analyzed the problem from our viewpoint of almost all newbies joining existing open source projects.\\

\noindent The choice for project and type of contribution was completely unconstrained, and while limited to twenty-four students we could gather a large variety of experiences ranging from hardcore open source software projects (e.g. bitcoin), to citizen science, to community wikis. From these heterogenous experiences, using first qualitative reporting complemented by a survey, we could identify and characterize most important issues enabling or on the contrary raising barriers against efficient open source project on boarding.\\ 

\noindent Our collective experience highlights the need for human-to-human communication: personal contact is an efficient outreach method and helps significantly guide participants into their first contributions. However, many of the projects we have joined lacked proper documentation that would facilitate joining. We openly ask whether some communities really desire newcomers for their project or at least, whether appropriate governance structures are in place to welcome new contributors. Our results, and experience writing this report, highlighted the prevalence and the importance benevolent dictatorship as the most prominent model of management in open source projects. Besides onboarding documentation and welcoming community structure,  broad visibility on tasks ownership was seen as an helpful in the joining process. We also analyzed business models and funding : our sample included mix of projects funded by academic grants, donations or corporation supported. These funding options have a clear impact on the choice of licensing: academic and and donations supported open source projects used more restrictive licenses, such as GNU GPL and BSD while more business oriented projects used most prominently Apache license. On the contrary to previous studies \cite{belenzon2012} and obviously with much less points, we found that license is not an important factor for project choice. We surmise, however, that our sample of 24 UC Berkeley students is highly biased, and poorly represents the broader variety of motivations for joining open source projects. However, it might give a reasonably reliable view of the intentions of graduate students when joining open source projects.\\

\noindent While limited, this field study suggests some interesting directions for future research. Students joining few years old projects was significantly different than for those having joined more established projects.  We would like to see future research directed towards illuminating the differences in joining experiences between participants who take part in projects of different  development stages. \\

\noindent Finally, this collective report also reflects the experience gained from a first attempt to teach open collaboration and peer-production at UC Berkeley in a way that respects as much as possible the spirit of open source. Pooling knowledge and resources for the sake of achieving such a collective work in a limited time has been a great challenge and an opportunity to learn about the somewhat frustrating misalignment of incentives that necessarily create tensions with the ultimate goal of achieving and delivering a collective work. We leave a report, which is less than perfect, but precisely strongly emphasizes on the current limitations of this experience. We hope that others will take on the challenge to reproduce and improve the class \cite{classweb2013}, and possibly build further knowledge on the intricate relationship between joining open source projects and the way communities manage onboarding of newbies. Accumulating practical knowledge on the onboarding process is critical for both the increasing interest by enthusiasts in joining projects, and for the long-term sustainability of open source communities.


