\section{Conclusion}

This report originates form UC Berkeley School of Information course on peer-production and open collaboration \cite{classweb2013}, with the main goals to get ``hands-on" experience of joining and contributing to open collaboration projects, and to understand the underpinning incentive mechanisms and contribution dynamics. Joining open source projects is not as simple as one might expect : multiple technical, incentive, social and emotional issues as well as imperfect communication tools  might prevent efficient joining. 

As students (and instructors) the experience acquired in the class guided our field exploration of joining scripts and how they are influenced by organization structures. Acquiring and maintaining a critical mass of developers as well as solving governance problems are among the most critical organizational issues in open collaboration \cite{} This report sheds further light on these problems and deepens the analysis by surveying down-to-earth aspects of our collectively acquired experience.

The choice for project and type of contribution was completely unconstrained, and while limited to twenty-four students we could gather a large variety of experiences ranging from hardcore open source software projects (e.g. bitcoin) to citizen science to community wiki (i.e. Oakland Wiki). From these heterogenous experiences we could identify and characterize most important issues enabling or on the contrary raising barriers against efficient open source project onboarding. Using first qualitative reporting complemented by a survey, we have analyzed these issues.

Our collective experience highlights the need for human-to-human communication : personal contact is an efficient outreach method and helps significantly guide participants into their first contributions. However, many of the projects we have joined lacked proper documentation to help joining. We openly ask whether some communities really desire newcomers for their project or at least, whether appropriate governance structures are in place. ur sample highlighted that benevolent dictatorship is the most prominent model of management. However, the management was supported by task management: visibility and ownership of tasks was seen helpful indicator on the community and also helping in the joining process. We also analyzed business models and funding. Our sample included mix of projects funded by academic grants, donations or corporation supported. These funding options have a clear impact on the choice of licensing: academic and and donations supported open source projects used more restrictive licenses, such as GNU GPL and BSD while more business oriented projects used most prominently Apache license. On the contrary to previous studies \cite{belenzon2012} and obviously with much less points ( 24 students)  we found that license is not an important factor for project choice.

While limited, this field study suggests some interesting directions for future research. Students joining few years old projects was significantly different than for those having joined more established projects.  We would like to see future research directed towards illuminating the differences in joining experiences between participants who take part in projects of different ages and stages. 

Finally, this collective report also reflects the experience gained from this first attempt to teach open collaboration and peer-production at UC Berkeley in a way that ``respects" as much as possible the spirit of open source. Pooling knowledge and resources for the sake of achieving such a collective work in a limited time has been a great challenge and an opportunity to learn about the somewhat frustrating misalignment of incentives. We leave a report which is less than perfect, and hope that others will take on the challenge to reproduce the class \cite{classweb2013}, and possibly build further knowledge on this first reported experience.

