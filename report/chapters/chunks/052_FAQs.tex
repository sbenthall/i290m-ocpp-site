\mysubsubsection{FAQs and Onboarding Documentation}
Around 50\% of projects have no actual FAQ or onboarding documentation, so there may exist opportunities to help newcomers by creating this type of content. Only 8\% have an initial contact designated. That may be due to varying organization structures, possibly even reflecting a deliberate choice to have the entry point be distributed and/or in flux (adaptive to varying needs). Both the contact people and the dispatch of inquiries may isimply be handled on ad hoc, informal, voluntary bases. Since members probably do have forum identities or other contact points, presumably, newbies can take initiative and contact them.

\begin{figure}[ht!]
\centering
\includegraphics[width=120mm]{chapters/img/documentation.png}
\caption{How well are projects documented }
\label{overflow}
\end{figure}

Onboarding practice documentation may exhibit the potential for qualitative analysis, such as text analysis. (Perhaps just raising the issue  to project leads would raise interesting questions about governance and organizational processes). {\bf There might be correlations between FAQ/onboarding info available and the org structure/governance of the actual projects}. A summary follows of the projects and the types of onboarding guidance they provide, which often indicate the types of newcomer being targeted:

\begin{itemize}
\item {\bf Hypothesis:} Development-centric (may be sufficient for their needs)
\item {\bf Peerlibrary:} Several different ways to contribute are listed and described. Since a member of our own group wrote it, it may not be a coincidence that this is one of the best all-purpose joining guides.
\item {\bf Mozilla PDF:} Mentions that all ideas are open (also,is mainly bug, feature, and development-oriented)
\item {\bf Courtlistener:} Has a great ``ways to help" page
\item {\bf Facebook React:} Developer and bug-centric, but also apologizes that they are still working on both improving transparency and creating a smoother, easier flow for contributors.
\item {\bf Chromium:} Developer-centric
\item {\bf Civi-CRM:} no FAQ, but does express and embrace a``just dive in" culture, which may be sufficient
\item {\bf Geonode:} The ``about us" page is a bit FAQ-like, with some info on getting started, contact info, etc.
\item {\bf OaklandWiki (a site of the Localwiki project):} the product is a wiki which requires no login to contribute content. One can immediatley add or edit content with no account. There is no apparent moderation, so they may someday have problems with quality or spam.
\item {\bf Mifos: } Complex options, but anyone can create an account. The ways to contribute are mainly technical, and some contribution screens require login, but many volunteering opportunities and tasks can be found in their volunteer sections-- the site was a bit fragmented. Some confusion may arise from the availability of both ``contributor" and ``volunteer" sections and subsections. The technical and non-technical areas had various options and areas. Presumably more options to participate become available after creating a free account for login access.
\end{itemize}

\noindent The other projects do not yet have any FAQ or onboarding documentation. Since there may be shared responsibility, or unclear responsibility, for marketing and community-building in these projects (these functions may be absorbed by members in decentralized fashion), it is sometimes not clear whether improving onboarding information would benefit the projects, or decisions have already been made that the current practices are sufficient. There was a striking variety of information that was available to help people with various backgrounds and skill levels join the communities. 




