\mysubsubsection{True Motivations for Open Source Contribution}
%{\it What do you want to get in return for your contribution?} 
The reason for asking about contributor motivation is to figure out if contributors are intrinsically motivated or if they want anything in return for volunteering their time. Time is money, and it would be naive to think that everybody is truly altruistic \cite{Algon2013cooperation}. The survey results seem to support this notion. Out of 24 responses, a {\it staggering}  50\% wanted more experience in open source communities. This should be taken with a grain of salt, however, because respondents are class peers. 25\% are looking for career advancement or nothing at all, while a few receive satisfaction, skills, societal benefit, and web development experience from contributing to open source projects. It is apparent that career advancement is a major reason for open source collaboration, as one might learn how to organize and run a lean company.\\

\noindent Strangely enough, 50\% of respondents wanted nothing at all, and only 2 sought academic recognition. The sample size is small and skewed by academic environment of respondents, so it would be interesting to extrapolate from a larger and more diverse group of respondents. The true motivation for joining, according to the survey, is to learn new skills and be part of a community, something that drives the average contributor to willingly volunteer their time. While they may seek nothing at all in terms of monetary compensation; the skills, connections, and societal benefit motivate respondents much more.
