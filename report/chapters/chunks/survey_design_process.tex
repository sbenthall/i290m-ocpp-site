\subsubsection{Survey Design Process}
To complement our qualitative understanding of the {\it joining process} we decided to design and run a survey. Each of us proposed a survey question and posted it on a Google Form \footnote{\url{https://docs.google.com/forms/d/1KnkSkM3f_QBRQeYfUXpyXspSqavjcClKS2DoWAz1fyE/viewform}}. We then categorized questions by assigning tags : communication, governance, contributions, social networks, joining (to be completed). One or more tags could be assigned to one question and has allowed to draw a bipartite network of relationships between questions (c.f. Figure, to be completed).

Once the first survey design was completed each of us took it. We debriefed the result in class and found several flaws that we described in a separate paragraph. However, the advantage of designing a survey in a collective way helped ensure that most questions would be relevant to most of us.

This is a starting point of course and in the future {\bf this survey could be iteratively improved}

\subsubparagraph{FAQ's and Onboarding Documentation}

Around 50\% of our projects have no actual FAQ or onboarding document, so an opportunity to help newcomers likely exists. Since only 8\% have an initial contact designated, there may be varying org structures, or a deliberate choice to have the entry point be distributed and in flux (according to needs). I do wonder if there could be some voluntary assignments. However, since members probably have forum or other contact points, presumably, newbies can take initiative and contact them.

I was exploring the qualitative potential of onboarding practices- text analysis may help (as well as some follow-up questions. Perhaps just raising the issue itself would capture attention of some project people.). Perhaps there is a connection between FAQ/onboarding info available and the org structure/governance of the actual projects.(Of course, the product and/or nature of the project is also relevant). My tentative summary:

Hypothesis: Development-centric (may be sufficient for their needs)
Peerlibrary: Ways to contribute are listed, described (Nice!). Hmm. Since Mitar wrote it, it may not be a coincidence that this is one of the best joining guides.
Mozilla PDF: Mentions all ideas are open (also, mainly bug, feature, and development-oriented)
Courtlistener: Great "ways to help" page
Facebook React: Developer and bug-centric, apologizes that easy flow for contribution is still being smoothed out (as well as transparency!)
Chromium: Developer-centric
Civic-CRM: no FAQ, but does express and embrace a  "just dive in" culture, which may be sufficient
Geonode: The about us page is a bit FAQ-like, with some info on getting started, contact, etc.
OTHER PROJECTS: No FAQ/onboarding document (yet?)


