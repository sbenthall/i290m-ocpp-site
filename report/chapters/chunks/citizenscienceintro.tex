\subsubsection{Citizen Science Intro}

As scientific data has become more open and accessible, either because of Federal regulations or changes in research culture, new collaboration opportunities were created among research teams. The citizen science movement is an example of crowd-sourced data collection or data processing that allows volunteers to access and collaborate in order to facilitate large-scale discovery. Volunteers, or citizen scientists, collaborate on a project or platform and have a variety of expertise levels. Although forms of citizen science have been seen throughout history, recent permutations of crowd-sourced science was illustrated in projects like SETI \footnote{http://www.seti.org/about-us}, where volunteers could donate their personal computer's processing power to help study life in the universe. 

Most citizen science projects are hosted through academic organizations like universities, or non-profits focused on a specific research or issue area. Citizen scientists volunteer their time and expertise to projects of interest to them, and are usually not compensated for their research, unlike an academic scientific team would be. Often the crowd-sourced, citizen science projects may only represent a small fraction of an otherwise larger research area. Depending upon the level of community or volunteer engagement, a web-portal is often hosted to facilitate the collection of observations or data processing.