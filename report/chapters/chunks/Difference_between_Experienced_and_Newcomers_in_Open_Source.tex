\subsubsection{The Difference between Experienced and Newcomers in Open Source}

To start my analysis, I divided the total number of responses into two categories based on how they answered the question : ``Prior to this project, did you have any experience in the past with open source projects?"
As a newcomer to open source communities myself, I wanted to analyze to {\bf see if there were any discernable differences or interesting trends between newcomers and ``veterans" of open source communities.} 
I started with the responses for how long did the individual lurk before making an attempt at reaching out to the community. We can see that in contrast to the ``No" group, the ``Yes" group has 2/11 or 18\% of people who didn't even lurk for a single day. On the  ``No" side, we can see that not  a single individual spent 0 days lurking and all of them had to spend at least 1-3 days lurking prior to joining.  This could indicate a stronger willingness to immediately dive in to a new Open Source community by the people who have had prior experience with them and a heightened shyness exhibited by the group with no experience with them before. The distribution for the YES  group is very disjoint when compared to the No group. For 71\%(5/7) of the Yes Group, it took them either more than a month or immediately to enter the OS community. This could indicate either {\it an extreme willingness or elongated hesitation that is not seen in the No Group} since there are no individuals that took 0 days and not as much of a frequency(20\% vs 43\%) of individuals that took more than 29+ days.  Statistically, {\it the distribution of the ``Yes" group is much more bimodal} with the peaks being at 0 days and 29+ days compared to the No group. 
 	Another interesting division between the two groups was their responses for the question ``what do  I plan on getting out of this contribution" For the Yes Group, the majority of them wanted not only experience in OS software but some sort of other reward like academic publication, career advancement, etc. However, this is decidedly false for the No Group where the majority wanted only experience in OS software or even nothing at all. This could be from the fact that because the Yes Group has contributed to Open Source communities before, they would need an extra incentive to join a new community while the No Group is simply excited to be part of this endeavor that they have never been on before. I think with further data, we could see if the requirement or desire for external incentives would increase linearly with the number of OS communities one would participate in.

Lastly, I thought it would be interesting to test if the two different groups would have differences in the roles that they play within their respective open source communities. I found that the number of roles a particular user from the Yes Group is significantly more than the number of roles a user from the No Group would take. This is to be expected as it is the No Group's first time. Other than that difference, the No Group had less developers and less documenters than the Yes Group but the same number of UX/Designers, Testers, and Analysts as the Yes Group. 
