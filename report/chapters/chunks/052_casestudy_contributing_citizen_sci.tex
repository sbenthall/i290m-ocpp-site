\subsection{Case Study -- Contributing to Citizen Science}

Citizen science is based upon crowd sourcing aspects of science projects to the public. Crowd sourced projects may require some pre-requisite knowledge or none at all. Citizen science allows average “citizens” to engage and contribute to scientific research by volunteering their time to conduct tasks laid out by a scientific team. \\
 
\noindent Zooniverse, a platform of citizen science projects, began as a collection of researchers who wanted to open up their own research projects for crowd-sourced contributions and data processing and evolved to be a host organization for other collaborator’s research projects. Zooniverse is an intricate user interface and platform created by the Citizen Science Alliance (CSA). CSA governs the activities and direction of Zooniverse and related projects, and works with the scientific community to develop new proposals for projects. \\
 
\noindent Citizen science projects, like those hosted on Zooniverse, span many scientific disciplines, are hosted on a platform that intentionally has a very low barrier to entry. This means that very little skill, pre-requisite knowledge, or time commitment is necessary to contribute to individual projects. Introductions or primers to the task being crowd-sourced are featured at the beginning of an observation module (usually for one specific project) and citizen scientists learn to do a task within minutes. A tutorial runs volunteers through common encounters, and then is available for assistance or frequently asked questions at any time. Creating these platforms for data collection from the volunteer citizen scientists requires much of the financial and personnel commitment from Zooniverse employees stationed at universities. \\
 
\noindent Joining is designed to be easy thanks to well-designed platforms like those on Zooniverse projects. Volunteers mostly use forums to discuss trends seen (which resulted in the a discovery of a new galaxy on Zoo Galaxy), ask questions to other citizen scientists or the scientific team, or post cool observations seen. Being a citizen scientist appears to be seen as more of an identity than becoming a member of a community like other open projects. Contributions can be made to a citizen science projects like those hosted on Zooniverse by contributing data processing or observations for the projects, or by opening up a research project of interest for contributions by other citizen scientists. There are also many opportunities to contribute to the academic literature and study of “open science,” so it is also possible to contribute to these projects by studying the larger movements in scientific research and the impact these changes are having on modern research infrastructure. \\
 
\noindent Since contributions can be made at any time, some projects may be completed within a short time span or may take a long amount of time. This variance is inherent with the decision to crowd-source a research project. The notion of “free loaders” does not really apply since it does not matter if a single person commits 5 observations or 500, so long as the project gets enough participants to move the project forward. Currently, there are no studies of if “productive bursts” occur on certain projects or how and what influences impact productivity on some projects.
